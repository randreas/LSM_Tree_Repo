\documentclass[sigconf]{acmart}
% ===== Don't touch this :) ======
\settopmatter{printacmref=false} % Removes citation information below abstract
\renewcommand\footnotetextcopyrightpermission[1]{} % removes footnote with conference information in first column
\renewcommand\footnotetextauthorsaddresses[1]{}
\pagestyle{plain} % removes running headers
% ================================

\usepackage{graphicx}


\begin{document}
\title{CS561 : Data Systems Architectures, a Template}

\author{Andy Huynh}
\affiliation{%
    \institution{Boston University}
}
\author{Juhyoung Mun}
\affiliation{%
    \institution{Boston University}
}
\author{Manos Athanassoulis}
\affiliation{%
    \institution{Boston University}
}

\begin{abstract}
    Abstract here! 
\end{abstract}

\maketitle

% =============================================================================
\section{Introduction}
% =============================================================================

\subsection{Motivation}

Every paper always comes with a motivation. Explain why this research is of interest
to the community.

\subsection{Problem Statement}

You \textbf{must} define your problem statement. This can be boiled down to a 
question you are answering or concept you are exploring. You don't need to go into
the nitty gritty details of your problem like variable definitions, but this is a
bit higher level.

\subsection{Contributions}

Give your reader a small summary of what exactly you 

You been reviewing a lot of papers this semester, take a look at what separated
the good ones from the great ones and analyze what how they introduced their
problem.


% =============================================================================
\section{Background}
% =============================================================================
Recommended work belong here! Would highly suggest you add any time of relevant
background information pertaining to your project (e.g. previous research)

If you need to explain with equations, \ref{eq:bf_fp} is an example of false positive 
rate of a Bloom filter~\cite{Bloom}.
\begin{equation}
    f_p = \Big(1-\Big(1-\frac{1}{m}\Big)^{kn}\Big)^{k} \approx \Big(1-e^{kn/m} \Big)^{k}
    \label{eq:bf_fp}
 \end{equation}

% =============================================================================
\section{Architecture}
% =============================================================================
This is where your papers will differ depending on what type of project you
have. At this point everything is extremely flexible and depends on how you
want to structure your paper.

\subsection{TemplateDB}

An example is the LSM project, generally you would want to start explaining
your system architecture here. Additional features may also belong here. If
you're benchmarking RocksDB you may end up explaining some of the experimental
set ups (assuming you've defined the problem at hand!)


\subsection{Solution Design}

Maybe you're doing a research project that tackles on a particular problem. You
may want to start introducing a design solution here that explains how you will
tackle said problem.

\subsection{Benchmark Explanations}

If you're doing some sort of quantitative analysis you may want your math to 
start being defined and analyzed here.


% =============================================================================
\section{Results}
% =============================================================================
Graphs are basically a requirement here!

\begin{figure}[ht]
    \includegraphics[width=3cm]{example-image-a}
    \caption{Example figure here}
    \label{fig:filler}
\end{figure}

\begin{table}[H]
    \caption{An example table}
	\label{tab:ex}
	\centering

	\begin{tabular}{c |c c c c c}
        \hline\hline
        & Col A & Col B \\
        \hline
        Row A & val 1 & val a \\
        Row B & val 2 & val b \\
        Row C & val 3 & val c \\
	\hline
	\end{tabular}
\end{table}

You can also reference your old figures (Figure \ref{fig:filler}) and tables 
(Table \ref{tab:ex})
% =============================================================================
\section{Conclusion}
% =============================================================================

Final thoughts here.

% reference
{
    \bibliographystyle{ACM-Reference-Format}
    \bibliography{biblio}
} 

\end{document}
