\documentclass[sigconf]{acmart}
% ===== Don't touch this :) ======
\settopmatter{printacmref=false} % Removes citation information below abstract
\renewcommand\footnotetextcopyrightpermission[1]{} % removes footnote with conference information in first column
\renewcommand\footnotetextauthorsaddresses[1]{}
\pagestyle{plain} % removes running headers
% ================================

\usepackage{graphicx}


\begin{document}
\title{CS561 : Implementation of LSM Tree}

\author{Richard Andreas}
\affiliation{%
    \institution{ra7296@bu.edu}
}
\author{Jingyu Su}
\affiliation{%
    \institution{Boston University}
}
\author{Xingkun Yin}
\affiliation{%
    \institution{Boston University}
}

\begin{abstract}
    In this paper, we explore the the design space of key-value stores that utilize the Log-Structured merge-tree (LSM-Tree) structure. More specifically, we have implemented a base version of the LSM-Tree that have been explored in previous work in order to get a better understanding of the basic knowledge of it. We decided to compare between performance of our design to another data structure (B+-tree), to see which data structure performs better based on different workloads and operations.
\end{abstract}

\maketitle

% =============================================================================
\section{Introduction}
% =============================================================================

\subsection{Motivation}

Log-Structured merge-tree (LSM-trees) are one of the most commonly used data structures for persistent storage of key-value entries. LSM-tree based storages are in use in several modern key-value stores including RocksDB at
Facebook, LevelDB and BigTable at Google, bLSM and cLSM at Yahoo!, Cassandra
and HBase at Apache. 

In addition to LSM Trees, B+ trees is another common data structure for key-value entries. A B+-tree is an index data structure that stores data pointers only in leaf nodes, and only pivot pivot pointers in the internal nodes. Additiaonlly, the leaf nodes are also linked to provide ordered access to the records,

\subsection{Problem Statement}

When choosing which data structure to use for indexing and searching, we typically would choose between these 2 data structures as they are widely known and have superior performance compared to the rest. However, this brings us to the next question, how do we pick between these two. In our paper, we will experiment on both data structures with various workloads and compare their performance in 4 operations: insert, delete, point scan, range scan. We believe our implementation of the LSM tree will be faster than that of B+ Trees.

\subsection{Contributions}

Give your reader a small summary of what exactly you 

You been reviewing a lot of papers this semester, take a look at what separated
the good ones from the great ones and analyze what how they introduced their
problem.


% =============================================================================
\section{Background}
% =============================================================================
Recommended work belong here! Would highly suggest you add any time of relevant
background information pertaining to your project (e.g. previous research)

If you need to explain with equations, \ref{eq:bf_fp} is an example of false positive 
rate of a Bloom filter~\cite{Bloom}.
\begin{equation}
    f_p = \Big(1-\Big(1-\frac{1}{m}\Big)^{kn}\Big)^{k} \approx \Big(1-e^{kn/m} \Big)^{k}
    \label{eq:bf_fp}
 \end{equation}

% =============================================================================
\section{Architecture}
% =============================================================================
This is where your papers will differ depending on what type of project you
have. At this point everything is extremely flexible and depends on how you
want to structure your paper.

\subsection{TemplateDB}

An example is the LSM project, generally you would want to start explaining
your system architecture here. Additional features may also belong here. If
you're benchmarking RocksDB you may end up explaining some of the experimental
set ups (assuming you've defined the problem at hand!)

\subsection{Fence Pointers}

\subsection{Range Scan}

\subsection{Testing}

\subsection{Benchmark Explanations}

If you're doing some sort of quantitative analysis you may want your math to 
start being defined and analyzed here.


% =============================================================================
\section{Results}
% =============================================================================
Graphs are basically a requirement here!

\begin{figure}[ht]
    \includegraphics[width=3cm]{example-image-a}
    \caption{Example figure here}
    \label{fig:filler}
\end{figure}

\begin{table}[H]
    \caption{An example table}
    \label{tab:ex}
    \centering

    \begin{tabular}{c |c c c c c}
        \hline\hline
        & Col A & Col B \\
        \hline
        Row A & val 1 & val a \\
        Row B & val 2 & val b \\
        Row C & val 3 & val c \\
    \hline
    \end{tabular}
\end{table}

You can also reference your old figures (Figure \ref{fig:filler}) and tables 
(Table \ref{tab:ex})
% =============================================================================
\section{Future work to be done}
% =============================================================================

% =============================================================================
\section{Conclusion}
% =============================================================================

Final thoughts here.

% reference
{
    \bibliographystyle{ACM-Reference-Format}
    \bibliography{biblio}

[1] Chen Luo, Michael J. Carey. LSM-based storage techniques: a survey. VLDB J.29(1): 393-418 (2020)

[2] Niv Dayan, Manos Athanassoulis, Stratos Idreos. Monkey: Optimal Navigable
Key-Value Store. SIGMOD Conference 2017: 79-94

[3] Patrick E. O'Neil, Edward Cheng, Dieter Gawlick, Elizabeth J. O'Neil. The LogStructured Merge-Tree (LSM-Tree). Acta Inf. 33(4): 351-385 (1996)
} 

\end{document}
